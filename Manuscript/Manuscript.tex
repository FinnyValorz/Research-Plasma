\documentclass{article}
\usepackage[backend=biber,style=phys]{biblatex}
\addbibresource{references.bib}
\usepackage{amsmath,amsthm,amssymb,amsfonts, mathtools, braket, cancel, bm, xcolor}
\usepackage[margin=1in]{geometry}

\newtheorem{theorem}{Theorem}
\newtheorem{lemma}{Lemma}
\newtheorem{corollary}{Corollary}
\newtheorem{remark}{Remark}
\newtheorem{definition}{Definition}

\newcommand{\para}{\parallel}
\newcommand{\lam}{\lambda}
\newcommand{\om}{\omega}
\newcommand{\gam}{\gamma}
\newcommand{\ep}{\epsilon}
\newcommand{\np}{\nabla_\perp}
\newcommand{\apo}{\ ^{\prime} \!}
\newcommand{\p}{\partial}
\newcommand{\ext}{\mathop{}\!\mathrm{d}}
\newcommand{\til}[1]{\widetilde{ #1 }}
\newcommand{\deriv}[2]{\frac{\p #1}{\p #2}}
\newcommand{\st}{\sin\theta}
\newcommand{\ct}{\cos\theta}
\newcommand{\sphi}{\sin\phi}
\newcommand{\cphi}{\cos\phi}
\newcommand{\fr}{\frac{1+\ep q}{1+\ep\beta_\para}}
\newcommand{\frinv}{\frac{1+\ep\beta_\para}{1+\ep q}}

\newcommand{\pth} [1] {\left( #1 \right) }
\newcommand{\br} [1] {\left[ #1 \right] }
\newcommand{\bmat} [1] {\begin{bmatrix} #1 \end{bmatrix}}
\newcommand{\pmat} [1] {\begin{pmatrix} #1 \end{pmatrix}}




\title{Reduced MHD Manuscript}
\author{Finny Valorz}
\date{June 2024}
\begin{document}
\maketitle



1: MHD Eqns. 2: ND process and ordering. 3: Split para/perp. Alg from two. 4: Fast-Slow Attempt. 5: 2D Decomps and q. 6: Take 4 and change vars. 7: New wFSS. Appendix: Lemmas and Identities. 

\section{MHD Equations of Motion} 
The magnetohydrodynamic (MHD) equations describe the time evolution of a charged fluid's mass density $\rho$, velocity $\bm{v}$, and magnetic field $\bm{B}$ in some spatial domain $Q\subset \mathbb{R}^3$. In this analysis, we fix $Q=D^2\times S^1$, the solid 2-torus. We choose poloidal coordinates $x,y\in D^2$ on the cross-sectional discs, and toroidal coordinate $z \in S^1$. The poloidal diameter $a$ and outer toroidal circumference $L$ provide characteristic length scales for our system's dynamics. 

In addition to a solenoidal condition, $\nabla\cdot \bm{B}=0$, our system consists of a continuity equation \eqref{continuity}, momentum conservation \eqref{momentum}, and Faraday's law \eqref{faraday}:

\begin{align}
    \deriv{\rho}{t} &= -\nabla\cdot \pth{\rho \bm{v}} \label{continuity} \\ 
    \rho\deriv{\bm{v}}{t} &= \mu_0^{-1} \pth{\nabla\times \bm{B}} \times \bm{B} - \nabla p(\rho) - \rho \bm{v}\cdot\nabla \bm{v} \label{momentum} \\ 
    \deriv{\bm{B}}{t} &= \nabla \times \pth{\bm{v}\times \bm{B}}. \label{faraday}
\end{align}

We assume that the pressure $p=p(\rho)$ is a function only of density. Standard boundary conditions for this system of equations are
\begin{align*} 
    \bm{B}\cdot \bm{n} = 0 \quad \text{ and }\quad  \bm{v}\cdot \bm{n} = 0 \quad \text{ on }\quad \p Q,
\end{align*}
where $\bm{n}$ denotes the outward pointing unit normal on the surface $\p Q$. 



\section{Nondimensionalization and Ordering} 
The dependent variables in our system of PDEs are highly coupled to one another. We will give a formal description of these relationships by nondimensionalizing our MHD model and comparing the dimensionless parameters that remain. These parameters will characterize an ordering, an observation timescale, and various dynamical regimes. 

Because our domain factorizes, it is convenient to nondimensionalize our coordinates according to their characteristic length scales: 
\begin{align*}
    x=aX,\quad y=aY,\quad z=\frac{L}{2\pi}Z.
\end{align*}
This allows us to replace the explicit dependence on both $a$ and $L$ with just the aspect ratio $\ep = 2\pi a/L$. For example, if we multiply each equation by a factor of $a$, the standard gradient becomes $a\nabla = \np + \ep\bm{e}_Z\p_Z$, where $\np = \pth{\p_X, \p_Y, 0}$. In these coordinates, the divergenceless condition is    
\begin{align*}
    a\nabla\cdot\bm{B} = \np\cdot\bm{B}_\perp + \ep\p_Z B_Z = 0, \quad\text{or}\quad \np\cdot\bm{B}_\perp = -\ep\p_ZB_Z. 
\end{align*}

Many plasma experiments take place on large aspect-ratio toroidal domains, where the ratio $\ep = 2\pi a/L$ is small. In such experiments, the magnetic field is usually much stronger in the toroidal than the poloidal directions, and the variations are small. Density is also relatively constant when $\ep\to 0$. This behavior motivates us to examine the specific ordering where 
\begin{align*}
    \rho &= \rho_0 \pth{1+\ep r} \\
    \bm{v} &= v_0 \bm{\nu} \\ 
    \bm{B} &= B_0 \pmat{\ep\beta_x \\ \ep\beta_y \\ 1+\ep\beta_\para},
\end{align*}
with $r, \bm{\nu}, \beta_x, \beta_y,$ and $\beta_\para$ all $O\pth{1}$ in $\ep$. The constant coefficients $\rho_0, v_0,$ and $B_0$ represent values typical of a given experiment, which can be relatiely small or large. Time and pressure will be treated similarly, so that $t=t_0\tau$ and $p=p_0\pi(r)$ with $\tau,\pi(r) = O(1)$.

{\color{red} I should write it this way: 
\begin{align*}
    \rho &= \rho_0 \pth{1+\ep r} \\
    \bm{v} &= v_0 \bm{\nu} = v_0 \pmat{\nu_X \\ \nu_Y \\ \nu_Z} = v_0 \pmat{\bm{\nu}_\perp \\ \nu_Z} \\ 
    \bm{B} &= B_0 \bm{\beta} = B_0 \pmat{\ep\beta_X \\ \ep\beta_Y \\ 1+\ep\beta_\para} = B_0 \pmat{\ep\bm{\beta}_\perp \\ \beta_Z},
\end{align*}
Why do I have $\nu_\para$ instead of $\nu_Z$? Why does $\bm{\beta}_\perp$ not include the $\ep$?}

This scaling is similar to one taken by Strauss in 1975. In his scaling however, toroidal magnetic field fluctuations are assumed to be $O\pth{\ep^2}$, restricting the kinds of scenarios where this process can be applied. Despite density fluctuations being small for small $\ep$, he also allows variations at $O(1)$, so that 
\begin{align*}
    \rho &= \rho_0 r \\
    \bm{v} &= v_0 \bm{\nu} \\ 
    \bm{B} &= B_0 \pmat{\ep\beta_x \\ \ep\beta_y \\ 1+\ep^2\beta_\para}.
\end{align*}
These assumptions are updated in the current analysis. 

When we rewrite our nondimensionalized MHD system in the following sections, we will find the following dimensionless ratios:  
\begin{align*}
    \frac{t_0 v_0}{a} = 1, \qquad 
    \frac{p_0}{\mu_0^{-1}B_0^2} = \beta_0, \qquad 
    \frac{\rho_0\,v_0^2}{p_0} = M_0^2. 
\end{align*}
The first of these establishes a particular observation timescale, where we are zoomed in on dynamics in the poloidal plane. The second and third define the plasma-beta parameter, $\beta_0$, and Mach number, $M_0$, respectively. Different orderings of these quantities in $\ep$ correspond to different regimes, where plasma behavior is characteristically different. 


\section{Rescaled Evolution Equations}
\subsection{Continuity Equation}
In our new dimensionless coordinates, the continuity equation reads
\begin{align*}
    a\deriv{\rho}{t} &= -\np\cdot\pth{\rho\bm{v}_\perp} - \ep\p_Z\pth{\rho v_Z}. 
\end{align*}
\begin{align*}
    \ep\frac{a\rho_0}{t_0}\deriv{r}{\tau} &= -a\nabla\cdot\pth{\rho\bm{v}} \\ 
    &= -\rho_0v_0\np\cdot\pth{\pth{1+\ep r}\bm{\nu}_\perp} - \ep\rho_0v_0 \p_Z\pth{\pth{1+\ep r}\nu_\para} \\ 
    &= -\rho_0v_0 a\nabla\cdot\pth{\pth{1+\ep r} \bm{\nu}}, 
\end{align*}
or, identifying our scaling coefficients, 
\begin{align*} 
    \ep\deriv{r}{\tau} &= -\frac{t_0v_0}{a} a\nabla\cdot\pth{\pth{1+\ep r}\bm{\nu}}
    = -\np\cdot\pth{\pth{1+\ep r}\bm{\nu}_\perp} - \ep\p_Z\pth{\pth{1+\ep r}\nu_\para}. 
\end{align*}

\subsection{Momentum Conservation}
Equation \eqref{momentum} contains three terms which must be separated into their poloidal and toroidal components in order to identify dimensionless parameters.
\begin{align*}
    \frac{a\rho_0v_0}{t_0} \pth{1+\ep r} \deriv{\bm{\nu}}{\tau} &= \mu_0^{-1} \pth{a\nabla\times \bm{B}} \times \bm{B} - a\nabla p(\rho) - \rho \bm{v}\cdot a\nabla \bm{v}. 
\end{align*}
To isolate the evolution of $\bm{\nu}$, we will remove $\frac{\rho_0v_0}{t_0} \pth{1+\ep r}$ from both sides. 

For the first term, we make use of the identity $\pth{\nabla\times\bm{B}}\times\bm{B} = \bm{B}\cdot\nabla\bm{B} - \pth{\nabla\bm{B}}\cdot\bm{B}$ to get 
\begin{align*}
    \br{\pth{a\nabla\times\bm{B}}\times\bm{B}}_Z &= \bm{B}\cdot a\nabla B_Z - \ep\pth{\p_Z\bm{B}}\cdot\bm{B} \\ 
    &= \pth{\bm{B}_\perp\cdot\np B_Z + \ep B_Z\p_Z B_Z} - \pth{\ep\bm{B}_\perp\cdot\p_Z\bm{B}_\perp + \ep B_Z\p_ZB_Z} \\ 
    &= \bm{B}_\perp\cdot \np B_Z + \ep\bm{B}_\perp\cdot \p_Z\bm{B}_\perp \\ 
    &= B_0^2 \pth{\ep^2\bm{\beta}_\perp\cdot\np\beta_\para + \ep^3 \bm{\beta}_\perp\cdot\p_Z\bm{\beta}_\perp}  \quad\text{and}
\end{align*}
Note! $\beta_\perp\cdot\p_Z\beta_\perp = \frac{1}{2}\p_Z|\beta_\perp|^2$. Which expression is handier? 
\begin{align*}
    \br{\pth{a\nabla\times\bm{B}}\times\bm{B}}_\perp &= \bm{B}\cdot a\nabla\bm{B}_\perp - \pth{\np\bm{B}}\cdot\bm{B} \\ 
    &= \pth{\bm{B}_\perp\cdot\np\bm{B}_\perp + \ep B_Z\p_Z\bm{B}_\perp} - \pth{\pth{\np\bm{B}_\perp} \cdot\bm{B}_\perp + B_Z\np B_Z} \\ 
    &= \pth{\np\times\bm{B}_\perp}\times\bm{B}_\perp + B_Z\pth{\ep\p_Z\bm{B}_\perp - \np B_Z} \\ 
    &= \ep^2 B_0^2 \pth{\np\times\bm{\beta}_\perp}\times\bm{\beta}_\perp + B_0^2 \pth{1+\ep\beta_\para} \pth{\ep^2\p_Z\bm{\beta}_\perp - \ep\np\beta_\para}. 
\end{align*}

The second and third terms are just   
\begin{align*}
    a\nabla p &= p_0\pth{\np\pi + \ep\bm{e}_Z\p_Z\pi}, \quad\text{and} \\ 
    \rho\bm{v}\cdot a\nabla\bm{v} &= \rho_0v_0^2 \pth{1+\ep r} \pth{\bm{\nu}_\perp\cdot\np\bm{\nu} + \ep\nu_\para\p_Z\bm{\nu}}. 
\end{align*}
Altogether, removing a factor of $\pth{1+\ep r} \rho_0v_0/t_0$ from both sides, we have evolution equations for the dimensionless velocity fields $\nu_\para$ and $\bm{\nu}_\perp$:  
\begin{align*}
    \pth{1+\ep r} \deriv{\nu_\para}{\tau} &= \ep^2 \frac{t_0v_0}{a} \frac{\mu_0^{-1} B_0^2}{\rho_0v_0^2} \pth{\bm{\beta}_\perp\cdot\np\beta_\para + \bm{\beta}_\perp\cdot\p_Z\bm{\beta}_\perp} - \ep\frac{t_0v_0}{a}\frac{p_0}{\rho_0v_0^2} \p_Z\pi - \frac{t_0v_0}{a} \pth{1+\ep r} \pth{\bm{\nu}_\perp \cdot \np\nu_\para + \ep \nu_\para\p_Z\nu_\para} \\ 
    &= \ep^2\frac{1}{M_0^2\beta_0} \pth{\bm{\beta}_\perp\cdot\np\beta_\para + \ep\bm{\beta}_\perp\cdot\p_Z\bm{\beta}_\perp} - \ep \frac{1}{M_0^2\beta_0} \beta_0\p_Z\pi - \pth{1+\ep r} \pth{\bm{\nu}_\perp\cdot\np\nu_\para + \ep \nu_\para\p_Z\nu_\para}. 
\end{align*}
\begin{align*}
    \deriv{\bm{\nu}_\perp}{\tau} &= 
\end{align*}


\subsection{Faraday Induction}
\begin{align*} 
\deriv{B_z}{t} &= 
    \pth{B_\perp\cdot\np + B_z\p_z}v_z - \br{(v_\perp\cdot\np + v_z\p_z) + (\np\cdot v_\perp + \p_zv_z)} B_z \\
    &= B_\perp\cdot\np v_z - v_\perp\cdot\np B_z - v_z\p_zB_z - \np \cdot v_\perp B_z 
\end{align*}





\end{document}