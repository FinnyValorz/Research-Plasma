\documentclass{article}
\usepackage[backend=biber,style=phys]{biblatex}
\addbibresource{references.bib}
\usepackage{amsmath,amsthm,amssymb,amsfonts, mathtools, braket, cancel, bm, xcolor, verbatim}
\usepackage[margin=1in]{geometry}

\newtheorem{theorem}{Theorem}
\newtheorem{lemma}{Lemma}
\newtheorem{corollary}{Corollary}
\newtheorem{remark}{Remark}
\newtheorem{definition}{Definition}

\newcommand{\para}{\parallel}
\newcommand{\lam}{\lambda}
\newcommand{\om}{\omega}
\newcommand{\gam}{\gamma}
\newcommand{\ep}{\epsilon}
\newcommand{\np}{\nabla_\perp}
\newcommand{\lap}{\Delta_\perp}
\newcommand{\p}{\partial}
\newcommand{\ext}{\mathop{}\!\mathrm{d}}
\newcommand{\til}[1]{\widetilde{ #1 }}
\newcommand{\deriv}[2]{\frac{\p #1}{\p #2}}
\newcommand{\st}{\sin\theta}
\newcommand{\ct}{\cos\theta}
\newcommand{\sphi}{\sin\phi}
\newcommand{\cphi}{\cos\phi}
\newcommand{\fr}{\frac{1+\ep q}{1+\ep\beta_\para}}
\newcommand{\frinv}{\frac{1+\ep\beta_\para}{1+\ep q}}
\newcommand{\GN}{G_\perp^N}
\newcommand{\GD}{G_\perp^D}

\newcommand{\pth} [1] {\left( #1 \right) }
\newcommand{\br} [1] {\left[ #1 \right] }
\newcommand{\bmat} [1] {\begin{bmatrix} #1 \end{bmatrix}}
\newcommand{\pmat} [1] {\begin{pmatrix} #1 \end{pmatrix}}




\title{Reduced MHD Manuscript}
\author{Finny Valorz}
%\date{June 2024}
\begin{document}
\maketitle



1: MHD Eqns. 2: ND process and ordering. 3: Split para/perp. Alg from two. 4: Fast-Slow Attempt. 5: 2D Decomps and q. 6: Take EoMs and change vars. 7: New wFSS. Appendix: Lemmas and Identities. 

\section{MHD Equations of Motion} 
The magnetohydrodynamic (MHD) equations describe the time evolution of a charged fluid's mass density $\rho$, velocity $\bm{v}$, and magnetic field $\bm{B}$ in some spatial domain $Q\subset \mathbb{R}^3$. In this analysis, we fix $Q=D^2\times S^1$, the solid 2-torus. We choose poloidal coordinates $x,y\in D^2$ on the cross-sectional discs, and toroidal coordinate $z \in S^1$. The poloidal diameter $a$ and outer toroidal circumference $L$ provide characteristic length scales for our system's dynamics. 

In addition to a solenoidal condition, $\nabla\cdot \bm{B}=0$, our system consists of a continuity equation \eqref{continuity}, momentum conservation \eqref{momentum}, and Faraday's law \eqref{faraday}:

\begin{align}
    \deriv{\rho}{t} &= -\nabla\cdot \pth{\rho \bm{v}} \label{continuity} \\ 
    \rho\deriv{\bm{v}}{t} &= \mu_0^{-1} \pth{\nabla\times \bm{B}} \times \bm{B} - \nabla p(\rho) - \rho \bm{v}\cdot\nabla \bm{v} \label{momentum} \\ 
    \deriv{\bm{B}}{t} &= \nabla \times \pth{\bm{v}\times \bm{B}}. \label{faraday}
\end{align}

We assume that the pressure $p=p(\rho)$ is a function only of density. Standard boundary conditions for this system of equations are
\begin{align} 
    \bm{B}\cdot \bm{n} = 0 \quad \text{ and }\quad  \bm{v}\cdot \bm{n} = 0 \quad \text{ on }\quad \p Q,
\end{align}
where $\bm{n}$ denotes the outward pointing unit normal on the surface $\p Q$. 



\section{Nondimensionalization and Ordering} 
The dependent variables in our system of PDEs are highly coupled to one another. We will give a formal description of these relationships by nondimensionalizing our MHD model and comparing the dimensionless parameters that remain. These parameters will characterize an ordering, an observation timescale, and various dynamical regimes. 

Because our domain factorizes, it is convenient to nondimensionalize our coordinates according to their characteristic length scales: 
\begin{align*}
    x=aX,\quad y=aY,\quad z=\frac{L}{2\pi}Z.
\end{align*}
This split allows us to replace the explicit dependence on both $a$ and $L$ with the aspect ratio $\ep = 2\pi a/L$. For example, if we multiply each equation by a factor of $a$, the standard gradient becomes $a\nabla = \np + \ep\bm{e}_Z\p_Z$, where $\np = \pth{\p_X, \p_Y, 0}$. In these coordinates, the divergenceless condition is    
\begin{align}
    a\nabla\cdot\bm{B} = \np\cdot\bm{B}_\perp + \ep\p_Z B_Z = 0, \quad\text{or}\quad \np\cdot\bm{B}_\perp = -\ep\p_ZB_Z \quad\text{for}\quad \bm{B}_\perp = \pmat{B_X \\ B_Y \\ 0}.
\end{align}
Similarly, we use $\bm{v}_\perp = \pmat{v_X \\ v_Y \\ 0}$ to refer to the component of $\bm{v}$ perpendicular to $Z$. 

Many plasma experiments take place on large aspect-ratio toroidal domains, where the ratio $\ep = 2\pi a/L$ is small. In such experiments, the magnetic field is usually much stronger in the toroidal than the poloidal directions, and the variations are small. Density is also relatively constant when $\ep\to 0$. This behavior motivates us to examine the specific ordering where 
\begin{align}
    \rho &= \rho_0 \pth{1+\ep r}, \nonumber\\
    \bm{v} &= v_0\bm{\nu} = v_0 \pmat{\bm{\nu}_\perp \\ \nu_\para}, \\ 
    \bm{B} &= B_0 \pmat{\ep\bm{\beta}_\perp \\ 1 + \ep \beta_\para}, \nonumber
\end{align}
The seven nondimensional components of $r, \bm{\nu}_\perp = \pth{\nu_X, \nu_Y, 0}^T, \nu_\para, \bm{\beta}_\perp = \pth{\beta_X, \beta_Y, 0}^T,$ and $\beta_\para$ are written at $O\pth{1}$ in $\ep$ and describe leading order variations in our seven physical variables. The constant coefficients $\rho_0, v_0,$ and $B_0$ represent values typical of a given experiment, which can be relatiely small or large. Time and pressure will be treated similarly, so that $t=t_0\tau$ and $p=p_0\pi(r)$ with $\tau,\pi(r) = O(1)$.

This scaling is similar to one taken by Strauss in 1975. In his scaling however, toroidal magnetic field fluctuations are assumed to be $O\pth{\ep^2}$, restricting the kinds of scenarios where this process can be applied. Despite density fluctuations being small for small $\ep$, he also allows variations at $O(1)$, so that 
\begin{align*}
    \rho &= \rho_0 r, \\
    \bm{v} &= v_0 \pmat{\bm{\nu}_\perp \\ \nu_\para}, \\ 
    \bm{B} &= B_0 \pmat{\ep\bm{\beta}_\perp \\ 1+\ep^2\beta_\para}.
\end{align*}
These assumptions are updated in the current analysis. 

When we rewrite our nondimensionalized MHD system in the following sections, we will find the following dimensionless ratios:  
\begin{align*}
    \frac{t_0 v_0}{a} = \ep, \qquad 
    \frac{p_0}{\mu_0^{-1}B_0^2} = \beta_0, \qquad 
    \frac{\rho_0\,v_0^2}{p_0} = M_0^2. 
\end{align*}
The first of these establishes a particular observation timescale, where we are zoomed in on dynamics in the poloidal plane. The second and third define the plasma-beta parameter, $\beta_0$, and Mach number, $M_0$, respectively. Different orderings of these quantities in $\ep$ correspond to different dynamical regimes, where plasma behavior is characteristically different. 


\section{Rescaled Evolution Equations}
\subsection{Continuity Equation}
In our new dimensionless coordinates, the continuity equation reads
\begin{align*}
    \ep\frac{a\rho_0}{t_0}\deriv{r}{\tau} &= -a\nabla\cdot\pth{\rho\bm{v}} = -\rho_0v_0\np\cdot\pth{\pth{1+\ep r}\bm{\nu}_\perp} - \ep\rho_0v_0 \p_Z\pth{\pth{1+\ep r}\nu_\para}, 
\end{align*}
or, identifying our scaling coefficients, 
\begin{align} 
    \deriv{r}{\tau} &= -\np\cdot\pth{\pth{1+\ep r}\bm{\nu}_\perp} - \ep\p_Z\pth{\pth{1+\ep r}\nu_\para}. 
\end{align}

\subsection{Momentum Conservation}
Equation \eqref{momentum} contains three terms which must be separated into their poloidal and toroidal components in order to identify dimensionless parameters.
\begin{align*}
    \frac{a\rho_0v_0}{t_0} \pth{1+\ep r} \deriv{\bm{\nu}}{\tau} &= \mu_0^{-1} \pth{a\nabla\times \bm{B}} \times \bm{B} - a\nabla p(\rho) - \rho \bm{v}\cdot a\nabla \bm{v}. 
\end{align*}
For the first term, we make use of the identity $\pth{\nabla\times\bm{B}}\times\bm{B} = \bm{B}\cdot\nabla\bm{B} - \pth{\nabla\bm{B}}\cdot\bm{B}$ to get 
\begin{align*}
    \br{\pth{a\nabla\times\bm{B}}\times\bm{B}}_\perp &= \bm{B}\cdot a\nabla\bm{B}_\perp - \pth{\np\bm{B}}\cdot\bm{B} \\ 
    &= \pth{\bm{B}_\perp\cdot\np\bm{B}_\perp + \ep B_Z\p_Z\bm{B}_\perp} - \pth{\pth{\np\bm{B}_\perp} \cdot\bm{B}_\perp + B_Z\np B_Z} \\ 
    &= \pth{\np\times\bm{B}_\perp}\times\bm{B}_\perp + B_Z\pth{\ep\p_Z\bm{B}_\perp - \np B_Z} \\ 
    &= \ep^2 B_0^2 \pth{\np\times\bm{\beta}_\perp}\times\bm{\beta}_\perp + \ep B_0^2 \pth{1+\ep\beta_\para} \pth{\ep\p_Z\bm{\beta}_\perp - \np\beta_\para} \quad\text{and}
\end{align*}
\begin{align*}
    \br{\pth{a\nabla\times\bm{B}}\times\bm{B}}_\para &= \bm{B}\cdot a\nabla B_Z - \ep\pth{\p_Z\bm{B}}\cdot\bm{B} \\ 
    &= \pth{\bm{B}_\perp\cdot\np B_Z + \ep B_Z\p_Z B_Z} - \pth{\ep\bm{B}_\perp\cdot\p_Z\bm{B}_\perp + \ep B_Z\p_ZB_Z} \\ 
    &= \bm{B}_\perp\cdot \np B_Z - \ep\bm{B}_\perp\cdot \p_Z\bm{B}_\perp \\ 
    &= \ep^2B_0^2 \pth{\bm{\beta}_\perp\cdot\np\beta_\para - \ep\bm{\beta}_\perp\cdot\p_Z\bm{\beta}_\perp}.
\end{align*}
%Note! $\beta_\perp\cdot\p_Z\beta_\perp = \frac{1}{2}\p_Z|\beta_\perp|^2$. Which expression is handier? 
The second and third terms are just   
\begin{align*}
    a\nabla p &= p_0\pth{\np\pi + \ep\bm{e}_Z\p_Z\pi}, \quad\text{and} \\ 
    \rho\bm{v}\cdot a\nabla\bm{v} &= \rho_0v_0^2 \pth{1+\ep r} \pth{\bm{\nu}_\perp\cdot\np\bm{\nu} + \ep\nu_\para\p_Z\bm{\nu}}, 
\end{align*}
where derivatives of pressure are taken by expanding about $r=0$:
\begin{align*}
    \pi\pth{1+\ep r} &= \pi(1) + \pi'(1)\ep r + \frac{1}{2}\pi''(1)\pth{\ep r}^2 + \cdots = \sum_{n=1}^\infty \frac{\pi^{n}(1)}{n!}\pth{\ep r}^2.
\end{align*} 

Altogether, after canceling dimensional constants and identifying scaling parameters, we get evolution equations for $\bm{\nu}_\perp$ and $\nu_\para$:  
\begin{align}
    \pth{1+\ep r} \deriv{\bm{\nu}_\perp}{\tau} &= \ep^3 \frac{1}{M_0^2\beta_0} \pth{\np\times\bm{\beta}_\perp}\times\bm{\beta}_\perp + \ep^2\frac{1}{M_0^2\beta_0} \pth{1+\ep\beta_\para} \pth{\ep\p_Z\bm{\beta}_\perp - \np\beta_\para} \nonumber\\ 
    &- \ep\frac{1}{M_0^2\beta_0}\beta_0\np\pi - \ep\pth{1+\ep r} \pth{\bm{\nu}_\perp\cdot\np\bm{\nu}_\perp + \ep\nu_\para\p_Z\bm{\nu}_\perp}, \\ 
    \pth{1+\ep r} \deriv{\nu_\para}{\tau} &= \ep^3\frac{1}{M_0^2\beta_0} \pth{\bm{\beta}_\perp\cdot\np\beta_\para - \ep\bm{\beta}_\perp\cdot\p_Z\bm{\beta}_\perp} \nonumber\\ 
    &- \ep^2\frac{1}{M_0^2\beta_0}\beta_0 \p_Z\pi - \ep\pth{1+\ep r} \pth{\bm{\nu}_\perp\cdot\np\nu_\para + \ep \nu_\para\p_Z\nu_\para}.
\end{align}


\subsection{Faraday Induction}
Similarly, we will split Faraday's Law into components parallel and perpendicular to $Z$ using
\begin{align*}
    \deriv{\bm{B}}{t} &= \nabla\times\pth{\bm{v}\times\bm{B}} = \pth{\bm{B}\cdot\nabla} \bm{v} + \cancel{\pth{\nabla\cdot\bm{B}}} \bm{v} - \pth{\bm{v}\cdot\nabla} \bm{B} - \pth{\nabla\cdot\bm{v}} \bm{B}.
\end{align*}
This gives
\begin{align}\label{faradayperp}
a\deriv{\bm{B}_\perp}{t} = \ep\frac{aB_0}{t_0} \deriv{\bm{\beta}_\perp}{\tau} 
    &= \pth{\bm{B}_\perp\cdot\np + \ep B_Z\p_Z}\bm{v}_\perp - \pth{\bm{v}_\perp\cdot\np + \ep v_Z\p_Z}\bm{B}_\perp - \pth{\np\cdot\bm{v}_\perp + \ep \p_Zv_Z}\bm{B}_\perp \nonumber\\ 
    &= \np\times\pth{\bm{v}_\perp\times\bm{B}_\perp} + \ep\pth{\p_ZB_Z + B_Z\p_Z}\bm{v}_\perp - \ep\pth{\p_Zv_Z + v_Z\p_Z}\bm{B}_\perp \nonumber\\ 
    &= \ep\p_Z\pth{B_Z\bm{v}_\perp - v_Z\bm{B}_\perp} - \bm{e}_Z\times\np\pth{\bm{e}_Z\cdot\bm{v}_\perp\times\bm{B}_\perp} \\
    \Longrightarrow \deriv{\bm{\beta}_\perp}{\tau} &= \ep\p_Z\pth{\pth{1+\ep\beta_\para}\bm{\nu}_\perp - \ep\nu_\para\bm{\beta}_\perp} - \ep\bm{e}_Z\times\np\pth{\bm{e}_Z\cdot\bm{\nu}_\perp\times\bm{\beta}_\perp}
    %\np\times\pth{\bm{v}_\perp\times\bm{B}_\perp} &= \pth{B_\perp\cdot\np - \ep\p_ZB_Z} v_\perp - \pth{v_\perp\cdot\np + \np\cdot v_\perp} B_\perp
\end{align}
\begin{align} \label{faradaypara}
a\deriv{B_Z}{t} = \ep \frac{aB_0}{t_0} \deriv{\beta_\para}{\tau} &= 
    \pth{\bm{B}_\perp\cdot\np + \ep B_Z\p_Z}v_Z - \pth{\bm{v}_\perp\cdot\np + \ep v_Z\p_Z} B_Z - \pth{\np\cdot\bm{v}_\perp + \ep \p_Zv_Z} B_Z \nonumber \\
    &= \bm{B}_\perp\cdot\np v_Z - \bm{v}_\perp\cdot\np B_Z - \ep v_Z\p_ZB_Z - \pth{\np\cdot\bm{v}_\perp} B_Z \nonumber \\ 
    \Longrightarrow \deriv{\beta_\para}{\tau} &= \ep\bm{\beta}_\perp\cdot\np\nu_\para - \ep \bm{\nu}_\perp \cdot\np\beta_\para - \ep^2 \nu_\para\p_Z\beta_\para - \pth{\np\cdot\bm{\nu}_\perp} \pth{1+\ep\beta_\para}.
\end{align}
Line \eqref{faradayperp} was obtained by recognizing that $\bm{v}_\perp\times\bm{B}_\perp = \pth{\bm{e}_Z\cdot\bm{v}_\perp\times\bm{B}_\perp}\bm{e}_Z$ lies entirely in the $Z$ direction. 


\section{Fast-Slow Systems: First Attempt}
Notice that when $\ep$ is small, some of the time derivatives above are much smaller than others, so that those variables appear almost static relative to the others. The framework of fast-slow dynamical systems helps formalize this distinction. 

\begin{definition}[Fast-Slow System]
    A fast-slow dynamical system (with fast variable $y\in Y$ and slow variable $x\in X$) is one whose equations of motion depend smoothly on $\ep$,
    \begin{align*}
        \frac{dy}{d\tau} = \dot{y} &= f_\ep(x,y) = f_0 + \ep f_1 + \ep^2 f_2 + \cdots \\ 
        \frac{dx}{d\tau} = \dot{x} &= \ep g_\ep(x,y) = \ep \pth{g_0 + \ep g_1 + \ep^2 g_2 + \cdots},
    \end{align*}
    while satisfying the constraint:
    \begin{align} \label{constraint}
        D_yf_0(x,y): Y \rightarrow Y \ \text{is invertible whenever}\ f_0(x,y)=0.
    \end{align}
\end{definition}

The limit system $\dot{y}=f_0$, $\dot{x}=0$ is obtained by taking $\ep\to 0$. By (Lemma \ref{limitsystem}), only the limit system needs to obey constraint \eqref{constraint} to be fast-slow. 

In Section \ref{asymptotic corrections}, we will show that the trajectories of fast variables can be written as a function of slow variables according to the expansion, $y^*_\ep(x) = y^*_0 + \ep y^*_1 + \cdots$. In some cases $y^*_\ep(x)$ is not unique, and there is instead a family of possible fast trajectories. This occurs when $D_yf_0$ is only surjective instead of invertible. These systems are called ``weakly fast-slow." 

Our evolution equations have some terms that appear at different orders depending on $M_0$ and $\beta_0$. We will consider three regimes, each having $M_0^2\beta_0=\ep^2$. Then, to find our limit system, we take $\ep\to0$ and see which terms remain. Doing this in all three cases suggests a fast slow split where 
\begin{align*}
    x = (\nu_\parallel,\bm{\beta}_\perp),\quad y = (r,\bm{\nu}_\perp,\beta_\parallel).
\end{align*}


\subsection{Low-$\beta$ Scaling}
In low-$\beta$ scaling, $M_0^2 = 1$, $\beta_0 = \ep^2$, the limiting fast-variable evolution equations become
\begin{align*}
\p_\tau r &= -\np\cdot\bm{\nu}_\perp\\
\p_\tau \bm{\nu}_\perp & = -\np\beta_\para\\
\p_\tau \beta_\para & = -\np\cdot\bm{\nu}_\perp.
\end{align*}
This choice of dependent variables does not comprise a fast-slow split because $D_yf_0(x,y)$ is neither injective nor surjective, so it fails to satisfy the constraint \eqref{constraint}. The derivatives in
\begin{equation} \begin{split}
    D_yf_0(x,y) [\delta y] = \pth{\deriv{f_0^j}{y^i} \br{\delta y^i}} 
    &= \pmat{\deriv{}{r}f_0^r & \deriv{}{\bm{\nu}_\perp}f_0^r & \deriv{}{\beta_\para}f_0^r\\ \deriv{}{r}f_0^{\bm{\nu}_\perp} & \ddots & \vdots \\ \deriv{}{r}f_0^{\beta_\para} & \cdots & \deriv{}{\beta_\para}f_0^{\beta_\para}} 
    \bmat{\delta r \\ \delta \bm{\nu}_\perp \\ \delta \beta_\para} \\
    &= \pmat{0 & -\np \cdot & 0 \\ 0 & 0 & -\np \\ 0 & -\np \cdot & 0} 
    \bmat{\delta r \\ \delta \bm{\nu}_\perp \\ \delta \beta_\para}
    = \pmat{-\np \cdot \delta \bm{\nu}_\perp \\ -\np \delta \beta_\para \\ -\np \cdot \delta \bm{\nu}_\perp }.
\end{split} \end{equation}
are found using $\p f_0^j/\p y^i \br{\delta y^i} = \frac{d}{d\alpha} f_0^j \pth{y^i+ \alpha \delta y^i}$.  
\begin{proof} [Injectivity:] 
    A linear map is injective if and only if its kernel is $\{0\}$. Let $\delta y = \pth{\delta r,0,0}$ for some $\delta r\neq 0$. $\delta y \in\ker D_yf_0$, so $D_yf_0$ is not injective.
\end{proof}
\begin{proof} [***Surjectivity:] 
    Given any $\delta\bar{y}\in Y$, we are unable to construct a $\delta y\in Y$ for which $D_yf_0[\delta y] = \delta\bar{y}$.  $\delta r$
\end{proof}

Intuitively, a fast-slow system ought to change on the fast timescale with a change in one of its fast variables. The system above is degenerate in the fast variable $r$, which suggests there are hidden slow variables among these fast variables.


\subsection{High-$\beta$ Scaling}
In the high-$\beta$ scaling $M_0^2 = \ep$, $\beta_0 = \ep$, the limiting fast-variable evolution equations are identical to the low-$\beta$ scaling, and so do not comprise a fast-slow split:
\begin{align*}
\p_\tau r & = -\np\cdot\bm{\nu}_\perp\\
\p_\tau \bm{\nu}_\perp & = -\np\beta_\para\\
\p_\tau \beta_\para & = -\np\cdot\bm{\nu}_\perp.
\end{align*}

\subsection{Low-Flow Scaling}
In low-flow scaling, $M_0^2 = \epsilon^2$, $\beta_0 = 1$ (where $\beta_0$ is actually larger than in high-$\beta$), the limiting fast-variable evolution equations are
\begin{align*}
\p_\tau r &= -\np\cdot\bm{\nu}_\perp\\
\p_\tau \bm{\nu}_\perp & = -\np (\pi^\prime(1)r + \beta_\para)\\
\p_\tau \beta_\para & = -\np\cdot\bm{\nu}_\perp.
\end{align*}
This time, fast directional derivative looks like 
\begin{align} 
    D_yf_0(x,y) [\delta y] 
    &= \pmat{0 & -\np \cdot & 0 \\ -\pi'(1)\np & 0 & -\np \\ 0 & -\np \cdot & 0}
    \bmat{\delta r \\ \delta \bm{\nu}_\perp \\ \delta \beta_\para} = \pmat{-\np \cdot \delta \bm{\nu}_\perp \\ -\np \pth{\pi'(1)\delta r + \delta \beta_\para} \\ -\np \cdot \delta \bm{\nu}_\perp}.
\end{align}
Although this system is no longer degenerate in $r$, it still fails to satisfy condition (\ref{constraint}). 
\begin{proof} [Injectivity:]
    $\delta y+c\in\ker D_yf_0$ for any $\delta y=0$ and constant $c$. 
\end{proof}
\begin{proof} [Surjectivity:]
    ***
\end{proof}

In each regime, the fact that the apparent fast-slow split is not correct suggests there is a nicer set of dependent variables to describe ideal MHD that cleanly separates fast and slow dynamics. The following section is devoted to identifying such variables.


\section{Nicer Dependent Variables for MHD}
Given any planar vector field $\bm{w}_\perp$ on a domain $D$ diffeomorphic to the unit disc with $\bm{n}\cdot\bm{w}_\perp = 0$ on $\partial D$, there are unique scalar fields $\phi,\psi:D\rightarrow\mathbb{R}$ such that
\begin{align}
\bm{w}_\perp = \np\phi + \bm{e}_Z\times\np\psi,\quad \bm{n}\cdot\np\phi = 0\text{ on }\partial D,\quad \int_D\phi \,dxdy = 0,\quad \psi = 0\text{ on }\partial D.\label{potential_rep}
\end{align}
The scalars are determined from $\bm{w}_\perp$ by solving the pair of Poisson equations given by
\begin{align} \label{poisson}
    \lap\phi = \np\cdot\bm{w}_\perp, && \lap\psi = \bm{e}_Z\cdot \np\times\bm{w}_\perp,
\end{align}
subject to the boundary conditions listed in \eqref{potential_rep}. As described in the appendix (\ref{}), these equations can be inverted by the Green's operators $\GN:L^2\to\mathcal{N}$ and $\GD:L^2\to\mathcal{D}$, where $\mathcal{N}$ and $\mathcal{D}$ are spaces of $L^2$ functions with homogeneous Neumann and Dirichlet boundary conditions on the disc, respectively:
\begin{align*}
    \phi = \GN\lap\phi = \GN\pth{\np\cdot\bm{w}_\perp}, && \psi = \GD\lap\psi = \GD\pth{\bm{e}_Z\cdot\np\times\bm{w}_\perp}.
\end{align*}

For each fixed $Z$, these considerations apply to the vector fields $\bm{\nu}_\perp$ and $\bm{\beta}_\perp$. Thus, there are ($Z$-dependent) scalar fields $\phi,\psi,\Phi,\Psi$ such that
\begin{align*}
\bm{\nu}_\perp & = \np\phi + \bm{e}_Z\times \np\psi\\
\bm{\beta}_\perp & = \np\Phi + \bm{e}_Z\times \np\Psi,
\end{align*}
and
\begin{gather*}
\psi = \Psi = 0 \text{ on } \p D,\quad \bm{n}\cdot\np\phi = \bm{n}\cdot\np\Phi = 0 \text{ on } \p D,\quad \iint_D \phi\ dxdy = \iint_D \Phi\ dxdy = 0,
\end{gather*}
where $D$ denotes the poloidal cross section of the spatial domain $Q$. We name the spaces of functions with homogeneous Dirichlet and Neumann boundary conditions $\psi,\Psi\in\mathcal{D}$ and $\phi,\Phi\in\mathcal{N}$, respectively. We will exchange the dependent variables $\bm{\beta}_\perp,\bm{\nu}_\perp$ with the four scalars $\phi,\Phi,\psi,\Psi$.

Note that the divergenceless magnetic field constraint, $\nabla\cdot\bm{B} = \np\cdot\bm{\beta}_\perp + \ep\p_Z\beta_\para = \lap\Phi + \ep\p_Z\beta_\para = 0$ actually lets us ignore one degree of freedom in $\bm{\beta}_\perp$. Specifically, $\Phi = -\ep\GN\p_Z\beta_\para$ is determined uniquely by $\beta_\para$. It is still convenient to write some equations with $\bm{\beta}_\perp$ and $\Phi$ though. 


\subsection{New $r$ Scaling}
Since mass density $\rho$ is advected as a volume form and the magnetic field $\bm{B}$ is advected as a $2$-form, the ratio $\bm{B}/\rho$ is advected as a vector field. This follows from the following manipulations using Cartan calculus. Let $\varrho = \rho\,dxdydz$ denote the mass density volume form and $\beta = \iota_{\bm{B}}dxdydz$ the magnetic flux $2$-form. On the one hand
\begin{align*}
(\partial_t + \mathcal{L}_{\bm{u}})(\iota_{\bm{B}/\rho}\varrho) = \iota_{(\partial_t + \mathcal{L}_{\bm{u}})(\bm{B}/\rho)}\varrho,
\end{align*}
because $\varrho$ is advected by $\bm{u}$. On the other hand 
\begin{align*}
(\partial_t + \mathcal{L}_{\bm{u}})(\iota_{\bm{B}/\rho}\varrho) = (\partial_t + \mathcal{L}_{\bm{u}})(\beta) = 0,
\end{align*}
because $\beta$ is advected by $\bm{u}$. Combining the two results implies $(\partial_t + \mathcal{L}_{\bm{u}})(\bm{B}/\rho) = 0$, as claimed. Contracting this vector field advection law with $dz$ implies 
\begin{align} \label{Qevolution}
\partial_tQ + \bm{u}\cdot\nabla Q = Q\frac{\bm{B}}{B_z}\cdot\nabla u_z,
\end{align}
where we have used the expressions \eqref{material continuity} - \eqref{material faraday}, and where $Q = B_z/\rho$.

We will exchange the dependent variable $r$ with a nondimensional version of $Q$, defined according to 
\begin{equation} 
    Q = \frac{B_z}{\rho} = \frac{B_0}{\rho_0} \frac{1+\ep\beta_\para}{1+\ep r} = q_0 (1+\ep q), 
\end{equation}
where $q_0 = B_0/\rho_0$. That is, we'll have 
\begin{align} \label{qdefinition}
    q = \frac{1}{\ep} \pth{\frac{1+\ep\beta_\para}{1+\ep r} - 1} = \frac{\beta_\para - r}{1+\ep r}.
\end{align}


\subsection{Helpful Brackets}
In order to streamline computations, let us introduce the symmetric and antisymmetric brackets $\Braket{\cdot,\cdot}, [\cdot,\cdot]: C^\infty\times C^\infty\to C^\infty$ defined by 
\begin{align}
    \Braket{f,g} = \np f\cdot \np g && \text{and} && \br{f,g} = \bm{e}_z\cdot\pth{\np f \times \np g}. 
\end{align}
Each bracket is clearly linear, and so is compatible with our ordering scheme. These brackets help to simplify the particularly common quantities
\begin{align}
    \np f\cdot \bm{w}_\perp &= \Braket{f, \phi} - [f, \psi], \\ 
    \bm{e}_z \cdot (\np f \times\bm{w}_\perp) &= \Braket{f, \psi} + [f, \phi]. 
\end{align}


\section{New Evolution Equations}
Now, we have 6 new variables for representing our system: $\phi$, $\psi$, $\nu_\para$, $\Psi$, $\beta_\para$, and $q$. In this section, we will simply rewrite our new equations of motion and truncate terms of order $\ep^3$ and higher. 

Running tally of changes to make to EoMs: switch $\tau\to T$, formally set $M_0^2\beta_0=\ep^2$, remove $\Phi$, expand $\fr$ and $\pi\pth{\frinv}$, remove terms $O\pth{\geq\ep^3}$, fully expanded remaining terms. 

\subsection{$\phi$ Evolution}
According to \eqref{poisson}, the evolution of $\phi$ is given by the divergence of the $\bm{\nu}_\perp$'s evolution: $\np\cdot\p_\tau\bm{\nu}_\perp = \p_\tau\lap\phi = \lap\p_\tau\phi$. Replacing $r$ with $q$ and using $\np\times\bm{\beta}_\perp = \lap\Phi$, we have
\begin{align*}
    \deriv{\bm{\nu}_\perp}{\tau} 
    &= \fr\pth{\ep\lap\Psi\bm{e}_Z\times\bm{\beta}_\perp - \ep^{-1}\beta_0\np\pi\pth{\frinv}} \\
    &+ \pth{1+\ep q} \pth{\ep\p_Z\bm{\beta}_\perp - \np\beta_\para} - \ep\pth{\bm{\nu}_\perp\cdot\np\bm{\nu}_\perp + \ep\nu_\para\p_Z\bm{\nu}_\perp}.
\end{align*}
The quantity $\np\pi\pth{\frinv}$ is given as an $\ep$ expansion in the appendix (\ref{}). Using the brackets defined in the previous section and several identities derived in the appendix (\ref{}), the divergence of this expression becomes
\begin{align}
    \lap\deriv{\phi}{\tau} &= \np\pth{\fr}\cdot\pth{\ep\lap\Psi\bm{e}_Z\times\bm{\beta}_\perp - \ep^{-1}\beta_0\np\pi\pth{\frinv}} \nonumber\\
    &+ \fr\np\cdot\pth{\ep\lap\Psi\bm{e}_Z\times\bm{\beta}_\perp - \ep^{-1}\beta_0\np\pi\pth{\frinv}} \nonumber\\
    &+ \ep\np q\cdot\pth{\ep\p_Z\bm{\beta}_\perp - \np\beta_\para} + \pth{1+\ep q}\np\cdot\pth{\ep\p_Z\bm{\beta}_\perp - \np\beta_\para} \nonumber\\
    &- \ep\np\cdot\pth{\bm{\nu}_\perp\cdot\np\bm{\nu}_\perp + \ep\nu_\para\p_Z\bm{\nu}_\perp} \nonumber\\
    &= -\ep\lap\Psi\Braket{\fr, \Psi} - \ep\lap\Psi\br{\fr, \Phi} - \ep^{-1}\beta_0\Braket{\fr, \pi\pth{\frinv}} \nonumber\\ 
    &- \ep\fr\Braket{\lap\Psi, \Psi} - \ep\fr\br{\lap\Psi, \Phi} - \ep\fr\pth{\lap\Psi}^2 - \ep^{-1}\beta_0\fr\lap\pi\pth{\frinv} \nonumber\\
    &+ \ep^2\Braket{q, \p_Z\Phi} - \ep^2\br{q, \p_Z\Psi} - \ep\Braket{q,\beta_\para} + \ep\p_Z\lap\Phi + \ep^2q\p_Z\lap\Phi - \lap\beta_\para - \ep q\lap\beta_\para \nonumber\\ 
    &- \ep\np\np\phi:\np\np\phi - \ep\Braket{\lap\phi, \phi} - \ep\lap\br{\psi,\phi} - \ep\np\np\psi:\np\np\psi \nonumber\\
    &+ \ep\br{\lap\psi, \phi} + \ep\pth{\lap\psi}^2 - \ep^2\nu_\para \lap\p_Z\phi - \ep^2\Braket{\nu_\para, \p_Z\phi} + \ep^2\br{\nu_\para, \p_Z\psi}.
\end{align}

After replacing  $\Phi=-\ep\GN\p_Z\beta_\para$ and expanding $\fr$ in $\ep$ (\ref{appendix}), we can determine precisely which order each term contributes at (with the exception of two terms which depend on choice of $\beta_0$): 
\begin{align*}
    \lap\deriv{\phi}{\tau} &= - \beta_0\ep^{-1} \Braket{\fr, \pi\pth{\frinv}} - \beta_0\ep^{-1} \fr\lap\pi\pth{\frinv} \\
    &- \ep^2\lap\Psi \Braket{q,\Psi} + \ep^2\lap\Psi \Braket{\beta_\para,\Psi} - \ep\Braket{\lap\Psi, \Psi} - \ep^2q \Braket{\lap\Psi, \Psi} + \ep^2 \beta_\para\Braket{\lap\Psi, \Psi} \\
    &+ \ep^2\br{\lap\Psi, \GN\p_Z\beta_\para} -\ep\pth{\lap\Psi}^2 + \ep^2q\pth{\lap\Psi}^2 - \ep^2\beta_\para\pth{\lap\Psi}^2 \\
    &- \ep^2\br{q, \p_Z\Psi} - \ep\Braket{q,\beta_\para} - \ep^2\p^2_Z\beta_\para - \lap\beta_\para - \ep q\lap\beta_\para \\
    &- \ep^2\nu_\para \lap\p_Z\phi - \ep^2\Braket{\nu_\para, \p_Z\phi} + \ep^2\br{\nu_\para, \p_Z\psi} - \ep\np\np\phi:\np\np\phi - \ep\Braket{\lap\phi, \phi} \\
    &- \ep\lap\br{\psi,\phi} - \ep\np\np\psi:\np\np\psi + \ep\br{\lap\psi, \phi} + \ep\pth{\lap\psi}^2 + O\pth{\ep^3}.
\end{align*}
When we expand the first line in $\ep$ using (\ref{appendix}), we find several terms that cancel with one another. Up to $O(\beta_0\ep)$, the result is 
\begin{align*}
    - \beta_0\ep^{-1} &\Braket{\fr, \pi\pth{\frinv}} - \beta_0\ep^{-1} \fr\lap\pi\pth{\frinv} \\
    = &-\beta_0\pi'(1) \pth{\lap\beta_\para - \lap q} \\ 
    &- \beta_0\ep \pi''(1) \Bigl( \beta_\para\lap \beta_\para - \beta_\para\lap q - q\lap\beta_\para + q\lap q + \left|\np\beta_\para\right|^2 - 2\Braket{\beta_\para,q} + |\np q|^2 \Bigr) \\ 
    &- \beta_0\ep \pi'(1) \Bigl( q\lap q - 2\Braket{\beta_\para,q} + \left|\np q\right|^2 - \left|\np\beta_\para\right|^2 - \beta_\para\lap\beta_\para \Bigr) + O\pth{\beta_0\ep^2}.
\end{align*}
\begin{comment}\begin{align*}
    - \beta_0\ep^{-1} &\Braket{\fr, \pi\pth{\frinv}} - \beta_0\ep^{-1} \fr\lap\pi\pth{\frinv} = \\
    &-\beta_0\pi'(1) \lap\beta_\para + \beta_0\pi'(1) \lap q - \beta_0\ep\pi''(1) \beta_\para\lap \beta_\para \\ 
    &+ \beta_0\ep\pi''(1) \beta_\para\lap q + \beta_0\ep\pi''(1) q\lap\beta_\para - \beta_0\ep\pi''(1) q\lap q \\ 
    &- \beta_0\ep\pi''(1) \left|\np\beta_\para\right|^2 + 2\beta_0\ep\pi''(1) \Braket{\beta_\para,q} - \beta_0\ep\pi''(1) |\np q|^2 \\ 
    &- \beta_0\ep\pi'(1) q\lap q + 2\beta_0\ep\pi'(1) \Braket{\beta_\para,q} - \beta_0\ep\pi'(1) \left|\np q\right|^2 \\ 
    &+ \beta_0\ep\pi'(1) \left|\np\beta_\para\right|^2 + \beta_0\ep\pi'(1) \beta_\para\lap\beta_\para + O\pth{\beta_0\ep^2}. 
\end{align*}\end{comment}
Besides these, the only $O(1)$ contribution is $-\lap\beta_\para$. At $O(\ep)$, we have
\begin{align*}
    &- \ep\Braket{\lap\Psi, \Psi} - \ep\pth{\lap\Psi}^2 - \ep\Braket{q,\beta_\para} - \ep q\lap\beta_\para - \ep\np\np\phi:\np\np\phi \\
    &- \ep\Braket{\lap\phi, \phi} - \ep\lap\br{\psi,\phi} - \ep\np\np\psi:\np\np\psi + \ep\br{\lap\psi, \phi} + \ep\pth{\lap\psi}^2.
\end{align*}
At $O(\ep^2)$, we have
\begin{align*}
    &- \ep^2\lap\Psi \Braket{q,\Psi} + \ep^2\lap\Psi \Braket{\beta_\para,\Psi} - \ep^2q \Braket{\lap\Psi, \Psi} + \ep^2 \beta_\para\Braket{\lap\Psi, \Psi} \\ 
    &+ \ep^2\br{\lap\Psi, \GN\p_Z\beta_\para} + \ep^2q\pth{\lap\Psi}^2 - \ep^2\beta_\para\pth{\lap\Psi}^2 \\
    &- \ep^2\br{q, \p_Z\Psi} - \ep^2\p^2_Z\beta_\para - \ep^2\nu_\para \lap\p_Z\phi - \ep^2\Braket{\nu_\para, \p_Z\phi} + \ep^2\br{\nu_\para, \p_Z\psi},
\end{align*}
and so on for higher orders. 


\subsection{$\psi$-Evolution}
According to (\ref{poisson}), $\bm{e}_Z\cdot\np\times\p_\tau\bm{\nu}_\perp = \lap\p_\tau\psi$. Applying this to \eqref{nuperpevolution},
\begin{align*}
    \lap\deriv{\psi}{\tau} &= \bm{e}_Z\cdot\np\pth{\fr} \times\pth{\ep\lap\Psi\bm{e}_Z\times\bm{\beta}_\perp - \beta_0\ep^{-1}\np\pi\pth{\frinv}} \\
    &+ \fr\bm{e}_Z\cdot\np\times\pth{\ep\lap\Psi\bm{e}_Z\times\bm{\beta}_\perp - \beta_0\ep^{-1}\np\pi\pth{\frinv}} \\
    &+ \bm{e}_Z\cdot \np q \times \pth{\ep^2\p_Z\bm{\beta}_\perp - \ep\np\beta_\para} + \pth{1+\ep q} \bm{e}_Z\cdot \np\times \pth{\ep\p_Z\bm{\beta}_\perp - \np\beta_\para} \\
    &- \ep\bm{e}_Z\cdot\np\times\pth{\bm{\nu}_\perp\cdot\np\bm{\nu}_\perp + \ep\nu_\para\p_Z\bm{\nu}_\perp} \\
    &= \ep\lap\Psi\Braket{\fr, \Phi} - \ep\lap\Psi\br{\fr, \Psi} - \beta_0\ep^{-1}\br{\fr, \pi\pth{\frinv}} \\ 
    &+ \ep\fr\Braket{\lap\Psi, \Phi} - \ep\fr \br{\lap\Psi, \Psi} + \ep\fr\lap\Phi\lap\Psi \\
    &+ \ep^2\Braket{q, \p_Z\Psi} + \ep^2\br{q, \p_Z\Phi} -\ep\br{q, \beta_\para} + \ep\lap\p_Z\Psi + \ep^2q\lap\p_Z\Psi \\
    &- \ep\Braket{\lap\psi, \phi} + \ep\br{\lap\psi, \psi} - \ep\lap\phi\lap\psi - \ep^2\Braket{\nu_\para, \p_Z\psi} - \ep^2\br{\nu_\para, \p_Z\phi} - \ep^2\nu_\para\lap\p_Z\psi.
\end{align*}
After replacing $\Phi=-\ep\GN\p_Z\beta_\para$ and expanding $\fr$, 
\begin{align*}
    \lap\deriv{\psi}{\tau} &= - \beta_0\ep^{-1} \br{\fr, \pi\pth{\frinv}} \\
    &- \ep^2\lap\Psi\br{q,\Psi} + \ep^2\lap\Psi\br{\beta_\para,\Psi} - \ep^2\Braket{\lap\Psi, \GN\p_Z\beta_\para} - \ep\br{\lap\Psi, \Psi} \\
    &- \ep^2q\br{\lap\Psi, \Psi} + \ep^2\beta_\para\br{\lap\Psi, \Psi} - \ep^2\p_Z\beta_\para\lap\Psi + \ep^2\Braket{q, \p_Z\Psi} \\
    &- \ep\br{q, \beta_\para} + \ep\lap\p_Z\Psi +\ep^2q\lap\p_Z\Psi - \ep\Braket{\lap\psi, \phi} + \ep\br{\lap\psi, \psi} \\
    &- \ep\lap\phi\lap\psi - \ep^2\Braket{\nu_\para, \p_Z\psi} -\ep^2\br{\nu_\para, \p_Z\phi} - \ep^2\nu_\para\lap\p_Z\psi + O\pth{\ep^3}.
\end{align*}
The only term containing $\beta_0$ is
\begin{align*}
    &- \beta_0\ep^{-1}\br{\fr, \pi\pth{\frinv}} \\
    &= 
\end{align*}
Besides this, we have no $O(1)$ contributions. At $O(\ep)$, 
\begin{align*}
    - \ep\br{\lap\Psi, \Psi} - \ep\br{q, \beta_\para} + \ep\lap\p_Z\Psi - \ep\Braket{\lap\psi, \phi} + \ep\br{\lap\psi, \psi} - \ep\lap\phi\lap\psi.
\end{align*}
At $O(\ep^2)$, 
\begin{align*}
    &- \ep^2\lap\Psi\br{q,\Psi} + \ep^2\lap\Psi\br{\beta_\para,\Psi} - \ep^2\Braket{\lap\Psi, \GN\p_Z\beta_\para} - \ep^2q\br{\lap\Psi, \Psi} + \ep^2\beta_\para\br{\lap\Psi, \Psi} \\
    &- \ep^2\p_Z\beta_\para\lap\Psi + \ep^2\Braket{q, \p_Z\Psi} + \ep^2q\lap\p_Z\Psi - \ep^2\Braket{\nu_\para, \p_Z\psi} - \ep^2\br{\nu_\para, \p_Z\phi} - \ep^2\nu_\para\lap\p_Z\psi.
\end{align*}


\subsection{$\nu_\para$-Evolution}
\begin{align*}
    \deriv{\nu_\para}{\tau} &= - \beta_0\fr\p_Z\pi\pth{\frinv} - \ep\pth{\bm{\nu}_\perp\cdot\np\nu_\para + \ep \nu_\para\p_Z\nu_\para} \\
    &+ \fr\ep\pth{\bm{\beta}_\perp\cdot\np\beta_\para - \ep\bm{\beta}_\perp\cdot\p_Z\bm{\beta}_\perp} \\
    &= - \beta_0\fr\p_Z\pi\pth{\frinv} - \ep\pth{\Braket{\nu_\para, \phi} - \br{\nu_\para, \psi} + \ep\nu_\para\p_Z\nu_\para} \\
    &+ \fr\ep\pth{\Braket{\beta_\para, \Phi} - \br{\beta_\para, \Psi} - \ep\Braket{\Phi, \p_Z\Phi} - \ep\Braket{\Psi, \p_Z\Psi} - \ep\br{\p_Z\Psi, \Phi} - \ep\br{\Psi, \p_Z\Phi}} \\
    &= - \beta_0\fr\p_Z\pi\pth{\frinv} - \ep\Braket{\nu_\para, \phi} + \ep\br{\nu_\para, \psi} - \ep^2\nu_\para\p_Z\nu_\para \\
    &-\ep^2\Braket{\beta_\para, \GN\p_Z\beta_\para} - \ep\br{\beta_\para, \Psi} - \ep^2q\br{\beta_\para, \Psi} + \ep^2\beta_\para\br{\beta_\para,\Psi} - \ep^2\Braket{\Psi, \p_Z\Psi} + O\pth{\ep^3}.
\end{align*}
Again, only one $\beta_0$-dependent term, 
\begin{align*}
    - \beta_0\fr\p_Z\pi\pth{\frinv} &= 
\end{align*}
No $O(1)$ terms. $O(\ep)$ terms are 
\begin{align*}
    - \ep\Braket{\nu_\para, \phi} + \ep\br{\nu_\para, \psi} - \ep\br{\beta_\para, \Psi}.
\end{align*}
At $O(\ep^2)$, 
\begin{align*}
    - \ep^2\nu_\para\p_Z\nu_\para - \ep^2\Braket{\beta_\para, \GN\p_Z\beta_\para} - \ep^2q\br{\beta_\para, \Psi} + \ep^2\beta_\para\br{\beta_\para,\Psi} - \ep^2\Braket{\Psi, \p_Z\Psi}.
\end{align*}


\subsection{$\Psi$-Evolution}
To find the evolution for $\Psi$, we take $\bm{e}_Z\cdot\np\times\p_\tau\bm{\beta}_\perp = \lap\p_\tau\Psi$ as we did for $\psi$.
\begin{align*}
    \deriv{\bm{\beta}_\perp}{\tau} &= \ep\p_Z\pth{\pth{1+\ep\beta_\para}\bm{\nu}_\perp - \ep\nu_\para\bm{\beta}_\perp} - \ep\bm{e}_Z\times\np\pth{\bm{e}_Z\cdot\bm{\nu}_\perp\times\bm{\beta}_\perp}
\end{align*}
\begin{align*}
    \lap\deriv{\Psi}{\tau} &= \ep\bm{e}_Z\cdot\p_Z\pth{\ep\np\beta_\para\times\bm{\nu}_\perp + \pth{1+\ep\beta_\para}\np\times\bm{\nu}_\perp - \ep\np\nu_\para\times\bm{\beta}_\perp - \ep\nu_\para\np\times\bm{\beta}_\perp} \\
    &- \ep\bm{e}_Z\cdot\pth{\np\cdot\np\pth{\bm{e}_Z\cdot\bm{\nu}_\perp\times\bm{\beta}_\perp}}\bm{e}_Z \\
    &= \ep\p_Z\pth{\ep\Braket{\beta_\para, \psi} + \ep\br{\beta_\para, \phi} + \pth{1+\ep\beta_\para}\lap\psi - \ep\Braket{\nu_\para, \Psi} - \ep\br{\nu_\para, \Phi} - \ep\nu_\para\lap\Psi} \\ 
    &- \ep\lap\pth{\Braket{\phi, \Psi} - \Braket{\psi, \Phi} + \br{\phi, \Phi} + \br{\psi, \Psi}} \\
    &= \ep^2\p_Z\Braket{\beta_\para, \psi} + \ep^2\p_Z\br{\beta_\para, \phi} + \ep\p_Z\lap\psi + \ep^2\pth{\p_Z\beta_\para}\lap\psi + \ep^2\beta_\para\p_Z\lap\psi \\
    &- \ep^2\p_Z\Braket{\nu_\para, \Psi} + \ep^3\br{\nu_\para, \GN\p_Z\beta_\para} - \ep^2\pth{\p_Z\nu_\para}\lap\Psi - \ep^2\nu_\para\p_Z\lap\Psi \\
    &- \ep\lap\Braket{\phi, \Psi} - \ep^2\lap\Braket{\psi, \GN\p_Z\beta_\para} + \ep^2\lap\br{\phi, \GN\p_Z\beta_\para} - \ep\lap\br{\psi, \Psi}.
\end{align*}
No $O(1)$ contributions. At $O(\ep)$, 
\begin{align*}
    \ep\p_Z\lap\psi - \ep\lap\Braket{\phi, \Psi} - \ep\lap\br{\psi, \Psi}. 
\end{align*}
At $O\pth{\ep^2}$,
\begin{align*}
    &\ep^2\p_Z \Braket{\beta_\para, \psi} + \ep^2\p_Z\br{\beta_\para, \phi} + \ep^2\pth{\p_Z\beta_\para}\lap\psi + \ep^2\beta_\para\p_Z\lap\psi - \ep^2\p_Z\Braket{\nu_\para, \Psi} \\
    &- \ep^2\pth{\p_Z\nu_\para}\lap\Psi - \ep^2\nu_\para\p_Z\lap\Psi - \ep^2\lap\Braket{\psi, \GN\p_Z\beta_\para} + \ep^2\lap\br{\phi, \GN\p_Z\beta_\para}. 
\end{align*}
At $O\pth{\ep^3}$, we just have $\ep^3\GD\p_Z\br{\nu_\para, \GN\p_Z\beta_\para}$, and nothing at higher orders. 


\subsection{$\beta_\para$-Evolution}
\begin{align*}
    \deriv{\beta_\para}{\tau} &= \ep\bm{\beta}_\perp\cdot\np\nu_\para - \ep \bm{\nu}_\perp \cdot\np\beta_\para - \ep^2\nu_\para\p_Z\beta_\para - \pth{\np\cdot\bm{\nu}_\perp} \pth{1+\ep\beta_\para} \\
    &= - \ep^2\Braket{\nu_\para, \GN\p_Z\beta_\para} - \ep\br{\nu_\para, \Psi} - \ep\Braket{\beta_\para, \phi} + \ep\br{\beta_\para, \psi} - \ep^2\nu_\para\p_Z\beta_\para - \lap\phi - \ep\beta_\para\lap\phi.
\end{align*}
If we would like to recover an evolution equation for $\Phi$, we can do so by taking $\p_\tau\Phi = -\ep\GN\p_Z\p_\tau\beta_\para$ of above. This equation has the $O(1)$ contribution $-\lap\phi$. At $O(\ep)$, 
\begin{align*}
    -\ep\br{\nu_\para, \Psi} - \ep\Braket{\beta_\para, \phi} + \ep\br{\beta_\para, \psi} - \ep\beta_\para\lap\phi.
\end{align*}
At $O(\ep^2)$,
\begin{align*}
    -\ep^2\Braket{\nu_\para, \GN\p_Z\beta_\para} - \ep^2\nu_\para\p_Z\beta_\para.
\end{align*}
There are no higher order contributions.


\subsection{$q$-Evolution}
\begin{align} \label{qevolution} 
    \p_\tau q &= \frac{t_0}{\ep q_0} \p_tQ = \frac{t_0}{\ep a q_0} \br{Q\frac{\bm{B}}{B_z} \cdot a\nabla v_z - \bm{v}\cdot a\nabla Q} \nonumber\\ 
    &= \frac{t_0v_0}{a} \frac{1}{\ep q_0} \frac{Q}{1+\ep\beta_\para} \ep\pth{\bm{\beta}_\perp \cdot \np\nu_\para} 
    + \frac{t_0v_0}{a} \frac{1}{\ep q_0} \frac{Q}{1+\ep\beta_\para} \ep\pth{1+\ep\beta_\para} \p_Z\nu_\para
    - \frac{t_0v_0}{a} \bm{\nu}_\perp \cdot \np q - \frac{t_0v_0}{a}\ep\nu_\para \p_Z q \nonumber\\
    &= \ep\fr\bm{\beta}_\perp\cdot\np\nu_\para + \ep\pth{1+\ep q}\p_Z\nu_\para - \ep\pth{\bm{\nu}_\perp\cdot\np q + \ep\nu_\para\p_Z q} \\
    &= \ep\fr\Braket{\nu_\para, \Phi} - \ep\fr\br{\nu_\para, \Psi} + \ep\pth{1+\ep q} \p_Z\nu_\para - \ep\Braket{q, \phi} + \ep\br{q, \psi} - \ep^2\nu_\para\p_Zq \\
    &= - \ep^2\Braket{\nu_\para, \GN\p_Z\beta_\para} - \ep\br{\nu_\para, \Psi} - \ep q\br{\nu_\para, \Psi} + \ep\beta_\para\br{\nu_\para, \Psi} \\
    &+ \ep\p_Z\nu_\para + \ep^2q\p_Z\nu_\para - \ep\br{q, \phi} + \ep\br{q, \psi} - \ep^2\nu_\para\p_Zq + O\pth{\ep^3}.
\end{align}
This has nothing at $O(1)$, but has $O(\ep)$
\begin{align*}
    - \ep\br{\nu_\para, \Psi} - \ep q\br{\nu_\para, \Psi} + \ep\beta_\para\br{\nu_\para, \Psi} + \ep\p_Z\nu_\para - \ep\br{q, \phi} + \ep\br{q, \psi}.
\end{align*}
At $O\pth{\ep^2}$,
\begin{align*}
    - \ep^2\Braket{\nu_\para, \GN\p_Z\beta_\para} + \ep^2q\p_Z\nu_\para - \ep^2\nu_\para\p_Zq.
\end{align*}
So on for higher orders.



\section{New Fast-Slow System}
Now we will again check if our system is fast-slow by sending $\ep\to0$ and seeing whether the limit system obeys the constraint \ref{constraint}.

\subsection{Low-$\beta$ Scaling}
In low-$\beta$ scaling ($M_0^2=1$, $\beta_0=\ep^2$), we appear to have $y=\pth{\phi,\beta_\para}$ and $x=\pth{\psi, \nu_\para, \Psi, q}$, as the limit system is
\begin{align*}
    \lap\p_\tau\phi = -\lap\beta_\para, && \p_\tau\beta_\para = -\lap\phi,
\end{align*}
with $\p_\tau x=0$. Because $\phi$ has Neumann boundary conditions, $\bm{n}\cdot\np\phi=0$, we also have that $\p_\tau\bm{n}\cdot\np\phi = \bm{n}\cdot\np\p_\tau\phi = 0$. Thus, we are permitted to use the Green's function outlined in the Appendix \ref{appendix} to write $\p_\tau\phi = \GN\lap\p_\tau\phi = -\GN\lap\beta_\para$. 

The derivative of the limit system with respect to the fast variables is 
\begin{align}
    D_yf_0[\delta y] &= \pmat{0 & -\GN\lap \\ -\lap & 0} \bmat{\delta\phi \\ \delta\beta_\para} = \bmat{-\GN\lap\delta\beta_\para \\ -\lap\delta\phi} = \bmat{\delta\bar{\phi} \\ \delta\bar{\beta}_\para}, 
\end{align}
where $\delta\phi \in\mathcal{N}$. We are interested in whether the derivative is invertible, or only in/surjective. 
\begin{proof} [Injectivity:]
    A linear map is injective if and only if its kernel is $\{0\}$. The kernel of this operator consists of all $\delta y\in Y$ for whom $D_yf_0(x,y)[\delta y] = 0$, or 
    \begin{align}
    \bmat{-\GN\lap\delta\beta_\para \\ -\lap\delta\phi} &= 
    \bmat{H^N_{\delta\beta_\para} - \delta\beta_\para \\ -\lap\delta\phi} = 
    \bmat{0 \\ 0},
    \end{align}
    where $H^N_{\delta\beta_\para}\in\mathcal{N}$ is the unique harmonic function with the same Neumann boundary conditions as $\delta\beta_\para$. Because $\delta\phi\in\mathcal{N}$, we do have that $\delta\phi = \GN\lap\delta\phi = 0$. However any harmonic function $\delta\beta_\para$ satisfies $\delta\beta_\para = H^N_{\delta\beta_\para}$, so $\ker D_yf_0 = \Set{ \bmat{0 \\ \delta\beta_\para} | \lap \delta\beta_\para=0} \neq \{0\}$. Therefore, the map $D_yf_0$ is not injective. 
\end{proof}
However, it is surjective:
\begin{proof} [Surjectivity:]
    Given any $\delta\bar{y} \in Y$, there exists a $\delta y\in Y$ such that $D_yf_0[\delta y] = \delta\bar{y}$. For example, we find one particular solution by choosing $\delta\beta_\para\in\mathcal{N}$, so that  
    \begin{align}
    \bmat{H^N_{\delta\beta_\para} - \delta\beta_\para \\ -\lap\delta\phi} =& \bmat{-\delta\beta_\para \\ -\lap\delta\phi} =  \bmat{\delta\bar{\phi} \\ \delta\bar{\beta}_\para} \\ 
    \Longrightarrow & \bmat{\delta\phi \\ \delta\beta_\para} = \bmat{-\GN \delta\bar{\beta}_\para \\ -\delta\bar{\phi}}. 
    \end{align}
\end{proof}
Because $D_yf_0$ is actually a surjection, this system is weakly fast slow.


\subsection{High-$\beta$ Scaling}
In high-$\beta$ scaling ($M_0^2 = \ep$, $\beta_0=\ep$), the limit system is identical to that in low-$\beta$ scaling, and we find that the split $x=\pth{\psi,\nu_\para, \Psi, q}$ and $y=\pth{\phi,\beta_\para}$ is weakly fast-slow in exactly the same way.


\subsection{Low-Flow Scaling}
In the low-flow scaling, ($M_0^2=\ep^2$, $\beta_0=1$), the limit system changes: 
\begin{align*}
    \lap\p_\tau\phi = -\lap\pth{\pth{\pi'(1)+1}\beta_\para - \pi'(1)q}, && \p_\tau\beta_\para = -\lap\phi.
\end{align*}
Applying the Green's operator once more, we have $\p_\tau\phi = -\pth{\pi'(1)+1}\GN\lap\beta_\para + \pi'(1)\GN q$ and fast-slow split $y=\pth{\phi,\beta_\para}$, $x=\pth{\psi,\nu_\para, \Psi, q}$. The $q$ term is annihilated when we take $D_yf_0(x,y)$ and we get 
\begin{align}
    D_yf_0(x,y)[\delta y] = \pmat{0 & -[p'(1)+1] \GN \lap \\ -\lap & 0} \bmat{\delta \phi \\ \delta \beta_\para} = \bmat{-[p'(1)+1] \pth{\delta\beta_\para - H_{\delta\beta_\para}} \\ -\lap\delta\phi} = \bmat{\delta \bar{\phi} \\ \delta \bar{\beta}_\para}. 
\end{align}
\begin{proof} [Injectivity:]
    This has the same non-trivial kernel as before, so it is not injective
\end{proof}
\begin{proof} [Surjectivity:]
    However, we can still find solutions for any $\delta\bar{y}$, so it is surjective, and our system is weakly fast slow. For example, when $\delta\beta_\para\in\mathcal{N}$, we have 
    \begin{align}
    \bmat{\delta\phi \\ \delta\beta_\para} &= \bmat{-\GN \delta\bar{\beta}_\para \\ -\delta\bar{\phi} / [p'(1)+1]}.
    \end{align}
\end{proof}



\section{Asymptotic Corrections} \label{asymptotic corrections}
Our solution for the trajectory of the fast variables in the limit system describes dynamics on the slow manifold, $S_0$. We will now assume that the full solution depends smoothly on $\ep$ so that we can gradually deform $S_0$ into $S_\ep$. The corresponding dynamics on $S_\ep$ can be written as $y^*_\ep(x) = y^*_0(x) + \ep y^*_1(x) + \ep^2y^*_2(x) + \cdots$, and should satisfy the invariance equation: 
\begin{align} \label{invariance}
    \dot{y}_\ep^*(x) &= \ep Dy^*_\ep(x)[g_\ep(x,y^*_\ep(x))] = f_\ep(x,y^*_\ep(x)).
\end{align}
Note that we can Taylor expand each $f_k$ in $f_\ep = f_0 + \ep f_1 + \ep^2 f_2 + \cdots$ around $y_0^*$ to get
\begin{align*} 
    f_k(x,y_\ep^*) &= f_k(x,y_0^*) + D_yf_k(x,y_0^*)[y_\ep^*-y_0^*] + \frac{1}{2}D^2f_k(x,y_0^*)[y_\ep^*-y_0^*]^2 + \cdots \\ 
    &= f_k(x,y_0^*) + \ep D_yf_k(x,y_0^*)[y_1^*+\ep y_2^*+\cdots] + \frac{1}{2}\ep^2 D^2f_k(x,y_0^*)[y_1^*+\ep y_2^*+\cdots]^2 + \cdots.
\end{align*}
The same applies to $g_\ep = g_0 + \ep g_1 + \ep^2 g_2 + \cdots$. 

At $O(1)$, the invariance equation is just $0=f_0\pth{x,y^*_0(x)}$, which lets us solve for $y_0^*$. At $O(\ep)$,    
\begin{align*}
    Dy^*_0[g_0\pth{x,y_0^*}] = f_1(x,y_0^*) + D_yf_0(x,y_0^*)[y^*_1], \quad\text{or} \\ 
    y_1^*(x) = [D_yf_0(x,y_0^*)]^{-1} \pth{Dy_0^*[g_0(x,y_0^*)] - f_1(x,y_0^*)}.  
\end{align*}
The second order correction $y_2^*$ is given by the second order contributions to the invariance equation: 
\begin{align*}
    Dy_1^*[g_0(x,y_0^*)] + Dy_0^*[g_1(x,y_0^*) + D_yg_0(x,y_0^*)[y_1^*]] &= f_2(x,y_0^*) + D_yf_1(x,y_0^*)[y_1^*] + D_yf_0(x,y_0^*)[y_2^*] \\ 
    &+ \frac{1}{2}D^2f_0(x,y_0^*)[y_1^*]^2, \quad\text{or} 
\end{align*}
\begin{align*}
    y_2^*(x) = [D_yf_0(x,y_0^*)]^{-1} & \biggl( Dy_1^*[g_0(x,y_0^*)] + Dy_0^*[g_1(x,y_0^*) + D_yg_0(x,y_0^*)[y_1^*]] \\ 
    &- f_2(x,y_0^*) - D_yf_1(x,y_0^*)[y_1^*] - \frac{1}{2}D^2f_0(x,y_0^*)[y_1^*]^2 \biggr). 
\end{align*}


\subsection{Low-$\beta$ Scaling}
We've already used the low-$\beta$ ($M_0^2=1$, $\beta_0=\ep^2$) limit system to solve for the zeroth order condition
\begin{align}
    f_0\pth{x,y^*_0(x)} &= \bmat{-\GN\lap\beta_{\para\,0}^* \\ -\lap\phi_0^*} = \bmat{0 \\ 0}.
\end{align}
Because $\phi_0^*\in\mathcal{N}$ already, we have $\phi_0^* = \GN\lap\phi_0^* = 0$. Any harmonic function satisfies $\GN\lap\beta_{\para\,0}^* = H_{\beta_{\para\,0}^*} - \beta_{\para\,0}^* = 0$, so we have 
\begin{align} \label{y^*_0}
    y_0^*(x) &= \Set{\bmat{0 \\ \beta_{\para\,0}^*(x)} | \lap\beta_{\para\,0}^*=0}.
\end{align}

Now we're going to find $y^*_1(x)$ using 
\begin{align*}
    D_yf_0(x,y_0^*)[y^*_1] = Dy^*_0[g_0\pth{x,y_0^*}] - f_1(x,y_0^*).
\end{align*}
$f_1(x,y_0)$ has the $O(\ep)$ contributions
\begin{align*}
    \lap\p_\tau\phi: &- \ep\Braket{\lap\Psi, \Psi} - \ep\pth{\lap\Psi}^2 - \ep\Braket{q,\beta_\para} - \ep q\lap\beta_\para - \ep\np\np\phi:\np\np\phi \\
    &- \ep\Braket{\lap\phi, \phi} - \ep\lap\br{\psi,\phi} - \ep\np\np\psi:\np\np\psi + \ep\br{\lap\psi, \phi} + \ep\pth{\lap\psi}^2, \\
    \p_\tau\beta_\para: &-\ep\br{\nu_\para, \Psi} - \ep\Braket{\beta_\para, \phi} + \ep\br{\beta_\para, \psi} - \ep\beta_\para\lap\phi.
\end{align*}
$f_1\pth{x,y_0^*}$ is given by inserting \eqref{y^*_0} into the above expressions. That is, by setting $\phi_0^* = \lap\beta_{\para\,0}^* = 0$.
\begin{align*}
    \lap\p_\tau\phi_0^*: &- \ep\Braket{\lap\Psi, \Psi} - \ep\pth{\lap\Psi}^2 - \ep\Braket{q,\beta_\para} - \ep\np\np\psi:\np\np\psi + \ep\pth{\lap\psi}^2, \\
    \p_\tau\beta_{\para\,0}^*: &-\ep\br{\nu_\para, \Psi} + \ep\br{\beta_\para, \psi}.
\end{align*}
$g_0(x,y_0)$ has $O(\ep)$ contributions
\begin{align*}
    \lap\p_\tau\psi: &- \ep\br{\lap\Psi, \Psi} - \ep\br{q, \beta_\para} + \ep\lap\p_Z\Psi - \ep\Braket{\lap\psi, \phi} + \ep\br{\lap\psi, \psi} - \ep\lap\phi\lap\psi, \\
    \p_\tau\nu_\para: &- \ep\Braket{\nu_\para, \phi} + \ep\br{\nu_\para, \psi} - \ep\br{\beta_\para, \Psi}, \\
    \lap\p_\tau\Psi: &+ \ep\p_Z\lap\psi - \ep\lap\Braket{\phi, \Psi} - \ep\lap\br{\psi, \Psi}, \\
    \p_\tau q: &- \ep\br{\nu_\para, \Psi} - \ep q\br{\nu_\para, \Psi} + \ep\beta_\para\br{\nu_\para, \Psi} + \ep\p_Z\nu_\para - \ep\br{q, \phi} + \ep\br{q, \psi}.
\end{align*}
$g_0(x,y^*_0)$ is given by inserting \eqref{y^*_0} into the above expressions.
\begin{align*}
    \lap\p_\tau\psi_0^*: &- \ep\br{\lap\Psi, \Psi} - \ep\br{q, \beta_\para} + \ep\lap\p_Z\Psi + \ep\br{\lap\psi, \psi}, \\
    \p_\tau\nu_{\para\,0}^*: &+ \ep\br{\nu_\para, \psi} - \ep\br{\beta_\para, \Psi}, \\
    \lap\p_\tau\Psi_0^*: &+ \ep\p_Z\lap\psi - \ep\lap\br{\psi, \Psi}, \\
    \p_\tau q_0^*: &- \ep\br{\nu_\para, \Psi} - \ep q\br{\nu_\para, \Psi} + \ep\beta_\para\br{\nu_\para, \Psi} + \ep\p_Z\nu_\para + \ep\br{q, \psi}.
\end{align*}
Altogether,
\begin{align*}
D_yf_0(x,y_0^*)[y_1^*] &= Dy_0^*\br{g_0(x,y_0^*)} - f_1(x, y_0^*) \\
&= \left( \begin{array}{l}
    \multicolumn{1}{c}{0} \\
    \p_\psi\beta_{\para\,0}^*\GD \pth{- \br{\lap\Psi, \Psi} - \br{q, \beta_\para} + \lap\p_Z\Psi + \br{\lap\psi, \psi}} \\
    + \p_{\nu_\para}\beta_{\para\,0}^* \pth{\br{\nu_\para, \psi} - \br{\beta_\para, \Psi}}
    + \p_\Psi\beta_{\para\,0}^*\GD \pth{\p_Z\lap\psi - \lap\br{\psi, \Psi}} \\
    + \p_q\beta_{\para\,0}^* \pth{- \br{\nu_\para, \Psi} -  q\br{\nu_\para, \Psi} + \beta_\para\br{\nu_\para, \Psi} + \p_Z\nu_\para + \br{q, \psi}}
\end{array} \right) \\
&+ \pmat{\GN\pth{\Braket{\lap\Psi, \Psi} + \pth{\lap\Psi}^2 + \Braket{q,\beta_\para} + \np\np\psi:\np\np\psi - \pth{\lap\psi}^2} \\
\br{\nu_\para, \Psi} - \br{\beta_\para, \psi}}
\end{align*}
or,
\begin{align}
    y_1^* &= \pmat{ \\ }
\end{align}

How to solve $\lap\beta_\para = s$? Requires $\iint_D \lap\beta_\para\,dA = \iint_D \lap\til{\beta}_\para\,dA = \int_{\p D} \np\til{\beta}_\para\cdot\bm{n}\,dl = 0$. If $\lap\beta_\para = \lap\phi$ for $\phi\in\mathcal{N}$, then we get $\iint_D \lap\beta_\para\,dA = \iint_D \lap\phi\,dA = 0$.





\subsection{High-$\beta$ Scaling}

$f_1(x,y_0)$ has the $O(\ep)$ contributions 
\begin{align*}
    \lap \p_T\phi &= - \np\np\phi:\np\np\phi - \Braket{\lap\phi, \phi} - \lap\br{\psi,\phi} - \np\np\psi:\np\np\psi + \br{\lap\psi, \phi} + \pth{\lap\psi}^2 \\
        &+ \left|\np\beta_\para\right|^2 - \Braket{q,\beta_\para} + P'\pth{\frinv} \lap\beta_\para - P'\pth{\frinv} \lap q \\ 
        &+\beta_\para\lap\beta_\para - q\lap\beta_\para - \frac{1}{2}\lap \beta_\para^2 - \frac{1}{2}\lap |\bm{\beta}_\perp|^2 \\ 
        &+ \lap\p_Z\Phi + \np\np\Phi:\np\np\Phi + \Braket{\lap\Phi, \Phi} + \lap\br{\Psi,\Phi} \\ 
        &+ \np\np\Psi:\np\np\Psi - \br{\lap\Psi, \Phi} - \pth{\lap\Psi}^2, \quad\text{and} \\ 
    \p_T\beta_\para &= \Braket{\nu_\para, \Phi} - \br{\nu_\para, \Psi} - \Braket{\beta_\para, \phi} + \br{\beta_\para, \psi} - \beta_\para\lap\phi. 
\end{align*}
$f_1(x,y_0^*)$ is given by inserting \eqref{y^*_0} into the above expression. That is, by setting $\phi^*_0 = \lap\beta_{\para\,0}^* = 0$: 
\begin{align*}
    \lap \p_T\phi_0^* &= - \np\np\psi:\np\np\psi + \pth{\lap\psi}^2 \\ 
        &+ |\np\beta_{\para\,0}^*|^2 - \Braket{q,\beta_{\para\,0}^*} - P'\pth{\frinv} \lap q \\ 
        &- \frac{1}{2}\lap \beta_{\para\,0}^{*\,2} - \frac{1}{2}\lap |\bm{\beta}_\perp|^2 \\ 
        &+ \lap\p_Z\Phi + \np\np\Phi:\np\np\Phi + \Braket{\lap\Phi, \Phi} + \lap\br{\Psi,\Phi} \\ 
        &+ \np\np\Psi:\np\np\Psi - \br{\lap\Psi, \Phi} - \pth{\lap\Psi}^2, \quad \text{and} \\
    \p_T\beta_{\para\,0}^* &= 
    \Braket{\nu_\para, \Phi} - \br{\nu_\para, \Psi} + \br{\beta_{\para\,0}^*, \psi}. 
\end{align*}
$\p_T\phi^*_0\in\mathcal{N}$ still has homogeneous Neumann boundary conditions, so 

$O(\ep)$ contributions to $g_0(x,y)$ are 
\begin{align*}
    \lap\p_T\psi &= - \Braket{\lap\psi, \phi} + \br{\lap\psi, \psi} - \lap\phi\lap\psi - \br{q,\beta_\para} + \lap\p_Z\Psi \\ 
        &+ \Braket{\lap\Psi, \Phi} - \br{\lap\Psi, \Psi} + \lap\Phi\lap\Psi, \\ 
    \p_T\nu_\para &= - \Braket{\nu_\para, \phi} + \br{\nu_\para, \psi} - \p_Z\beta_\para + \p_Z\beta_\para + \Braket{\beta_\para,\Phi} -\br{\beta_\para, \Psi}, \\ 
    \lap \p_T\Phi &= \lap \p_Z\phi, \\ 
    \lap \p_T\Psi &= - \lap\pth{\Braket{\phi, \Psi} - \Braket{\psi, \Phi} + \br{\phi, \Phi} + \br{\psi, \Psi} - \p_Z\psi}, \quad\text{and} \\ 
    \p_Tq &= \Braket{\nu_\para, \Phi} - \br{\nu_\para, \Psi} - \Braket{q, \phi} + \br{q, \psi}, \text{\color{blue}good}
\end{align*}
which greatly simplify for $g_0(x,y_0^*(x))$: 
\begin{align*}
    \bmat{\p_T\psi \\ \p_T\nu_\para \\ \p_T\Phi \\ \p_T\Psi \\ \p_Tq} 
    &= \bmat{\p_Z\Psi + \GD\pth{ \br{\lap\psi, \psi} - \br{q,\beta_{\para\,0}^*} + \Braket{\lap\Psi, \Phi} - \br{\lap\Psi, \Psi} + \lap\Phi\lap\Psi} \\ 
    \br{\nu_\para, \psi} + \Braket{\beta_{\para\,0}^*,\Phi} - \br{\beta_{\para\,0}^*, \Psi} \\ 
    0 \\ 
    \p_Z\psi + \GD\lap\pth{\Braket{\psi, \Phi} - \br{\psi, \Psi}} \\ 
    \Braket{\nu_\para, \Phi} - \br{\nu_\para, \Psi} + \br{q, \psi}}, \text{\color{blue}good}
\end{align*}
***These can be simplified with the $\Braket{} \pm \br{}$ identities I have. 

Hey, notice $\br{\phi,\Phi}\in\mathcal{D}$, so that $\GD\lap\br{\phi,\Phi} = \br{\phi,\Phi}$. Proof: 
\begin{align*}
    \br{\phi,\Phi}\bm{n} &= \bm{e}_Z\times \br{\bm{n}\times \pth{\np\phi\times\np\Phi}} + \cancel{\pth{\bm{e}_Z\cdot\bm{n}}} \pth{\np\phi\times\np\Phi} \\ 
    &= \bm{e}_Z\times \br{\pth{\bm{n}\cdot\np\Phi}\np\phi - \pth{\bm{n}\cdot\np\phi}\np\Phi} = 0 \quad\text{on}\quad \p D. 
\end{align*}
I still can't figure out whether something similar works for $\Braket{\phi,\Phi}$, or for different entries. 

\subsection{Low-Flow Scaling}








\end{document}